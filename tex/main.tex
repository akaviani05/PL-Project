\documentclass[a4paper,12pt]{article}
\usepackage{HomeWorkTemplate}
\usepackage{circuitikz}
\usepackage{multicol}
\usepackage{placeins}
\usepackage[shortlabels]{enumitem}
\usepackage{hyperref}
\usepackage{tikz}
\usepackage{minted}
\usetikzlibrary{automata,positioning}
\usepackage{tikz-timing}
\usepackage{bookmark}
\usepackage{booktabs}
\usepackage{amsmath}
\usepackage{amssymb}
\usepackage{tcolorbox}
\usepackage{xepersian}
\usepackage{circuitikz}
% \usepackage[bidi=basic]{bidi} % Load bidi without RTLdocument option
\settextfont{XB Niloofar} % Persian font
\setlatintextfont{Times New Roman} % English font
\usetikzlibrary{arrows,automata}
\usetikzlibrary{circuits.logic.US}
\usepackage{changepage}
\usepackage{capt-of}
\newcounter{problemcounter}
\newcounter{subproblemcounter}
\newcounter{subsubproblemcounter}
\setcounter{problemcounter}{1}
\setcounter{subproblemcounter}{2}
\setcounter{subsubproblemcounter}{1}

\newcommand{\abjadd}[1]{%
  \ifcase#1
   \or الف
   \or ب
    \or ج
    \or د
    \or ه
    \or و
    \or ز
    \or ح
    \or ط
    \or ی
    \or ک
    \or ل
    \or م
    \or ن
  \fi
}

\newcommand{\harfii}[1]{\abjadd{\numexpr#1-1\relax}}
\newcommand{\problem}[1]
{
	\subsection*{
		سوال
		\arabic{problemcounter} 
		\stepcounter{problemcounter}
		\setcounter{subproblemcounter}{2}
		#1
	}
}

\newcommand{\subproblem}{
  % if subproblem counter is not 0, add \\
  \setcounter{subsubproblemcounter}{1}
  \ifnum\value{subproblemcounter}>2
    \\ \\
  \fi
  \textbf{\harfii{\value{subproblemcounter}})}\stepcounter{subproblemcounter}
}

\newcommand{\subsubproblem}{
  % start from 1 to i
    \\
  \textbf{\arabic{subsubproblemcounter})}\stepcounter{subsubproblemcounter}
}

\begin{document}
\handout
{زبان‌های برنامه‌سازی}
{دکتر ایزدی}
{نیم‌سال دوم 1403\lr{-}1404}
{اطلاعیه}
{}
{}
{فاز ۱ پروژه}
\section{توضیحات گرامر:}
گرامر کامل زبان ترکیبی از گرامر سی، کمی از گرامر پایتون و ساده‌سازی های بسیار برای راحتی پیاده سازی و نوشتن لکسر و پارسر است، 
برای مثال، در گرامر استفاده شده تنها از 
\lr{while}
برای دستورات حلقه استفاده شده و از ایف‌های تک خطی(بدون اسکوپ) پشتیبانی نشده.
\\
برای ساپورت کردن از آرایه، تصمیم به این شد که علاوه بر دیتا‌ تایپ‌های بیسیک مانند 
\lr{int} و \lr{float} از دیتاتایپی به نام دیتاتایپ لیست استفاده شود، 
که سینتکس تعریف لیست به صورت زیر است:
\begin{minted}{c++}
list<typename> name;

example
list<list<int>> a;
\end{minted}
در این زبان، تمام آرایه‌ها داینامیک سایز تعریف می‌شوند و آرایه با سایز ثابت در آن وجود ندارد، همچنین برای تورینگ کامپلیت بودن 
سینتکس زبان، علاوه بر دو استیتمنت ابتدایی 
\texttt{input}
و 
\texttt{output}
چهار دستور کار با لیست وجود دارد که عبارتند از
\texttt{push}, \texttt{pop}, \texttt{get}, \texttt{set}
که عملکرد آن‌ها از اسم آنها مشخص است، دقت کنید چون آرایه‌ها داینامیک سایز هستند 
و اجازه دستری به آنها توسط آپرند داده نشده، داشتن حداقل ۲ تا از این توابع یعنی 
\texttt{get}
و 
\texttt{push}
برای تورینگ‌کامپلیت بودن الزامیست، ۲ تابع دیگر هم برای راحتی به استیتمنت‌های از پیش تعریف شده زبان اضافه شدند.
\\
دقت کنید که تمامی ۶ دستور از پیش معین شده، با علامت 
\$
شروع می‌شوند.
\\
در این زبان، درست مثل پایتون می‌توان در هر اسکوپی تابع را تعریف کرد و تعریف توابع مانند سی پلاس پلاس به اسکوپ گلوبال محدود نیستند، همچنین توابع باید با سمیکالن 
پایان یابند، برای مثال:
\begin{minted}{c++}
  int a;
  $input(a);
  int factorial(int n) {
    if (n == 0) {
      return 1;
    } else {
      return n * factorial(n - 1);
    }
  };

  $output(factorial(a));
\end{minted}
\pagebreak
\section{گرامر کامل زبان:}
در ادامه گرامر کامل زبان را نوشته‌ایم:

\begin{align*}
\text{\lr{Spc}} &\rightarrow \text{\lr{Space}} \mid \text{\lr{Tab}} \mid \text{\lr{NewLine}} \\[0.5em]
\text{\lr{Program}} &\rightarrow \text{\lr{StatementSeq}} \\[0.5em]
\text{\lr{SimpleStatement}} &\rightarrow \text{\lr{VarDec}} \mid \text{\lr{FunDec}} \mid \text{\lr{Assignment}} \mid \text{\lr{Expression}} \mid \\
&\quad \text{\lr{BreakStatement}} \mid \text{\lr{ContinueStatement}} \mid \text{\lr{ReturnStatemnt}} \mid \text{\lr{PreDefinedStatement}} \\[0.5em]
\text{\lr{Statement}} &\rightarrow \text{\lr{SimpleStatement}}; \mid \text{\lr{IfStatement}} \mid \text{\lr{WhileStatement}} \\[0.5em]
\text{\lr{StatementSeq}} &\rightarrow \text{\lr{Statement}} \; \text{\lr{StatementSeq}} \mid \text{\lr{Scope}} \; \text{\lr{StatementSeq}} \mid \text{\lr{empty}} \\[0.5em]
\text{\lr{Scope}} &\rightarrow \{ \; \text{\lr{StatementSeq}} \; \} \\[0.5em]
\text{\lr{Expression}} &\rightarrow \text{\lr{Exp7}} \\[0.5em]
\text{\lr{IfStatement}} &\rightarrow \text{\lr{if}} \; ( \; \text{\lr{Expression}} \; ) \; \text{\lr{Scope}} \mid \\
&\quad \text{\lr{if}} \; ( \; \text{\lr{Expression}} \; ) \; \text{\lr{Scope}} \; \text{\lr{else}} \; \text{\lr{Scope}} \mid \\
&\quad \text{\lr{if}} \; ( \; \text{\lr{Expression}} \; ) \; \text{\lr{Scope}} \; \text{\lr{else}} \; \text{\lr{IfStatement}} \\[0.5em]
\text{\lr{WhileStatement}} &\rightarrow \text{\lr{while}} \; ( \; \text{\lr{Expression}} \; ) \; \text{\lr{Scope}} \\[0.5em]
\text{\lr{Word}} &\rightarrow \text{\lr{``}}(a\text{\lr{-}}z,A\text{\lr{-}}Z,\_)(a\text{\lr{-}}z,A\text{\lr{-}}Z,\_,0\text{\lr{-}}9)^*\text{\lr{''}} \\[0.5em]
\text{\lr{VarName}} &\rightarrow \text{\lr{Word}} \\[0.5em]
\text{\lr{FunName}} &\rightarrow \text{\lr{Word}} \\[0.5em]
\text{\lr{VarType}} &\rightarrow \text{\lr{int}} \mid \text{\lr{float}} \mid \text{\lr{string}} \mid \text{\lr{char}} \mid \text{\lr{bool}} \mid \text{\lr{list<VarType>}} \\[0.5em]
\text{\lr{VarTypeName}} &\rightarrow \text{\lr{VarType}} \; \text{\lr{Spc}} \; \text{\lr{VarName}} \\[0.5em]
\text{\lr{VarDec}} &\rightarrow \text{\lr{VarTypeName}} \mid \text{\lr{VarTypeName}} = \text{\lr{Expression}} \\[0.5em]
\text{\lr{VarSeq}} &\rightarrow \text{\lr{VarTypeName}} , \text{\lr{VarSeq}} \mid \text{\lr{VarTypeName}} \\[0.5em]
\text{\lr{FunDec}} &\rightarrow \text{\lr{VarType}} \; \text{\lr{Spc}} \; \text{\lr{FunName}} \; ( \; ) \; \text{\lr{Scope}} \mid \\
&\quad \text{\lr{VarType}} \; \text{\lr{Spc}} \; \text{\lr{FunName}} \; ( \; \text{\lr{VarSeq}} \; ) \; \text{\lr{Scope}} \\[0.5em]
\end{align*}
\begin{align*}
\text{\lr{ExpSeq}} &\rightarrow \text{\lr{Expression}} \mid \text{\lr{Expression}} , \text{\lr{ExpSeq}} \\[0.5em]
\text{\lr{FunCal}} &\rightarrow \text{\lr{FunName}} \; ( \; \text{\lr{ExpSeq}} \; ) \mid \text{\lr{FunName}} \; ( \; ) \\[0.5em]
\text{\lr{BreakStatement}} &\rightarrow \text{\lr{break}} \\[0.5em]
\text{\lr{ContinueStatement}} &\rightarrow \text{\lr{continue}} \\[0.5em]
\text{\lr{ReturnStatemnt}} &\rightarrow \text{\lr{return}} \mid \text{\lr{return}} \; \text{\lr{Spc}} \; \text{\lr{Expression}} \\[0.5em]
\text{\lr{Assignment}} &\rightarrow \text{\lr{VarName}} = \text{\lr{Expression}} \\[0.5em]
\text{\lr{Op1}} &\rightarrow ! \mid \sim \\[0.5em]
\text{\lr{Op2}} &\rightarrow * \mid / \mid \% \\[0.5em]
\text{\lr{Op3}} &\rightarrow + \mid - \\[0.5em]
\text{\lr{Op4}} &\rightarrow \textless= \mid \textgreater= \mid \textless \mid \textgreater \mid == \mid != \\[0.5em]
\text{\lr{Op5}} &\rightarrow \& \mid \text{\lr{\^{}}} \mid \text{\lr{|}} \\[0.5em]
\text{\lr{Op6}} &\rightarrow \&\& \\[0.5em]
\text{\lr{Op7}} &\rightarrow \text{\lr{||}} \\[0.5em]
\text{\lr{Exp0}} &\rightarrow \text{\lr{Atom}} \mid ( \; \text{\lr{Exp7}} \; ) \\[0.5em]
\text{\lr{Exp1}} &\rightarrow \text{\lr{Op1}} \; \text{\lr{Exp0}} \mid \text{\lr{Exp0}} \\[0.5em]
\text{\lr{Exp2}} &\rightarrow \text{\lr{Exp2}} \; \text{\lr{Op2}} \; \text{\lr{Exp1}} \mid \text{\lr{Exp1}} \\[0.5em]
\text{\lr{Exp3}} &\rightarrow \text{\lr{Exp3}} \; \text{\lr{Op3}} \; \text{\lr{Exp2}} \mid \text{\lr{Exp2}} \\[0.5em]
\text{\lr{Exp4}} &\rightarrow \text{\lr{Exp4}} \; \text{\lr{Op4}} \; \text{\lr{Exp3}} \mid \text{\lr{Exp3}} \\[0.5em]
\text{\lr{Exp5}} &\rightarrow \text{\lr{Exp5}} \; \text{\lr{Op5}} \; \text{\lr{Exp4}} \mid \text{\lr{Exp4}} \\[0.5em]
\text{\lr{Exp6}} &\rightarrow \text{\lr{Exp6}} \; \text{\lr{Op6}} \; \text{\lr{Exp5}} \mid \text{\lr{Exp5}} \\[0.5em]
\text{\lr{Exp7}} &\rightarrow \text{\lr{Exp7}} \; \text{\lr{Op7}} \; \text{\lr{Exp6}} \mid \text{\lr{Exp6}} \\[0.5em]
\text{\lr{Atom}} &\rightarrow \text{\lr{PreDefinedStatement}} \mid \text{\lr{FunCal}} \mid \text{\lr{VarName}} \mid \text{\lr{Value}} \\[0.5em]
\text{\lr{Value}} &\rightarrow \text{\lr{BoolVal}} \mid \text{\lr{IntVal}} \mid \text{\lr{FloatVal}} \mid \text{\lr{CharVal}} \mid \text{\lr{StrVal}} \\[0.5em]
\text{\lr{BoolVal}} &\rightarrow \text{\lr{false}} \mid \text{\lr{true}} \\[0.5em]
\text{\lr{IntVal}} &\rightarrow (+-)?(0\text{\lr{-}}9)^+ \\[0.5em]
\text{\lr{FloatVal}} &\rightarrow \text{\lr{IntVal}} \mid (+-)?(0\text{\lr{-}}9)^*\backslash.(0\text{\lr{-}}9)^+ \\[0.5em]
\text{\lr{CharVal}} &\rightarrow \text{\lr{'}}.'\text{\lr{'}} \mid \text{\lr{'}}\backslash\backslash.\text{\lr{'}} \\[0.5em]
\end{align*}
\begin{align*}
\text{\lr{StrVal}} &\rightarrow \text{\lr{""}} \mid \text{\lr{"}}.*(\text{\lr{non}}\ \backslash)\text{\lr{"}} \\[0.5em]
\text{\lr{PreDefinedStatement}} &\rightarrow \text{\lr{PrintStatement}} \mid \text{\lr{InputStatement}} \mid \text{\lr{GetStatement}} \mid \\
&\quad \text{\lr{SetStatement}} \mid \text{\lr{PushStatement}} \mid \text{\lr{PopStatement}} \\[0.5em]
\text{\lr{PrintStatement}} &\rightarrow \text{\lr{\$print}} \; ( \; \text{\lr{Expression}} \; ) \\[0.5em]
\text{\lr{InputStatement}} &\rightarrow \text{\lr{\$input}} \; ( \; \text{\lr{VarName}} \; ) \\[0.5em]
\text{\lr{GetStatement}} &\rightarrow \text{\lr{\$get}} \; ( \; \text{\lr{VarName}} , \text{\lr{Expression}} \; ) \\[0.5em]
\text{\lr{SetStatement}} &\rightarrow \text{\lr{\$set}} \; ( \; \text{\lr{VarName}} , \text{\lr{Expression}} , \text{\lr{Expression}} \; ) \\[0.5em]
\text{\lr{PushStatement}} &\rightarrow \text{\lr{\$push}} \; ( \; \text{\lr{VarName}} , \text{\lr{Expression}} \; ) \\[0.5em]
\text{\lr{PopStatement}} &\rightarrow \text{\lr{\$pop}} \; ( \; \text{\lr{VarName}} \; )
\end{align*}
\end{document}
